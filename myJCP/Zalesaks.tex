\subsection{Zalesak's test problem}
Solid body rotation of a notched disc is commonly used to test the advection capabilities of an interface capture solver. Zalesak firstly introduced this test \citep{zalesak1979fully} and its variants have been used extensively. In this particular test, the radius of the disc is $R=0.15$, the notch width is $W=0.06$ and the notch height is $H=0.25$. The center of the disc lies in $(x_0,y_0)=(0.5,0.75)$ in a domain of $1\times{1}$ that is discretized using rectangular meshes of size $100\times{100}$,$200\times{200}$,$400\times{400}$ figure (\ref{fig:hex0}). The rotation velocity is given by the following equation,
\begin{equation}\label{27}
\begin{split}
&u=-2\pi(y-0.5)
\\
&v=2\pi(x-0.5).
\end{split}
\end{equation}
At all the domain boundary, zero gradient condition is set for $\alpha$ and $\phi$. 

\subsubsection{structured meshes}
In figure (\ref{fig:MULESHEX},\ref{fig:ISOHEX},\ref{fig:CLSHEX}), the solutions of five combinations of time and mesh resolution in columns are obtained with MULES, isoAdvector, CLSAdvection. The effects of refining mesh resolution are investigated with fixed Courant Number, $Co=0.5$. With the finest mesh, the effects of reducing $Co$ from $0.5$ to $0.1$ are tested. Errors and Efficiency measures are displayed in table \ref{Tab:01}. Then we can get the following observations. All the algorithms are able to realize the volume conservation. But CLSAdvection method can cause slightly mass loss, which is confined at $0.01$. Especially, CLSAdvection can get the same sharpness as isoAdvector which is a sharp interface advection method and far better sharpness than the MULES method. Both isoAdvector and CLSAdvection method can make sure the void fraction $\alpha$ bounded at $[0,1]$. And MULES algorithm can cause a slight over flow. From table (\ref{Tab:01}), we can see the coupled VOF and level set method take the most time to finish the cases, which is caused by solving reinitialization equation (\ref{11}) and level set equation (\ref{10}). Besides, from table (\ref{Tab:01}), the influence of Courant number is that the smaller $Co$ means smaller time steps and the longer simulation time. Nevertheless, shrinking Courant number doesn't mean to improve the precision. Actually, the errors can be accumulated as the simulations process. From the figure (\ref{fig:CLSHEX}) and figure (\ref{fig:ISOHEX}), the shape of the notched disc is kept well at $Co=0.5$ and deforms more explicitly in the cases of $Co = 0.1$ and $Co = 0.2$. On the other hand, the results show that MULES method can cause numerical oscillations. As the meshes are densified, the oscillations become apparent, which can be explained by introducing high resolution convection scheme including vanLeer, SuperBee, QUICK and so on \cite{deshpande2012evaluating}.

\subsubsection{unstructured meshes}
We apply such algorithm on unstructured meshes and the initial shapes in figure (\ref{fig:poly0}) and figure (\ref{fig:trio0}). Figure (\ref{fig:MULESPoly}), (\ref{fig:ISOPoly}) and (\ref{fig:CLSPoly})show calculation results of MULES, isoAdvector, CLSAdvection after a circle in polygon meshes. And figure (\ref{fig:MULESTri}), (\ref{fig:ISOTri}) and (\ref{fig:CLSTri})show calculation results of MULES, isoAdvector, CLSAdvection after a circle in triangle meshes. The $Co$ numbers are set as $0.1$ and $0.5$, while the meshes have three different resolutions with the cell number 51967 and 203965. The errors and calculation times are showed in table \ref{Tab:02} and \ref{Tab:03}. From the figures and tables, we can make such conclusions as follows. First, the basic function of coupled level set and VOF method can be realized in unstructured meshes. The cell cut algorithm can be applied in polygon and triangle meshes. Second, the CLSAdvection can get the same sharpness effect as the sharp interface method, isoAdvector. MULES method can cause the interface smeared. After all, CLSAdvection inherits the idea of geometric volume of fluid method and explicitly reconstruct the interface. Third, polygon meshes are proved to have the better interface capturing capability than triangle meshes. Apparently, structured meshes undoubtedly have the best calculation characteristic. However, the nature of widely applied CFD softwares like OpenFOAM$^{\textregistered}$ and Fluent$^{\textregistered}$ requires the algorithm should be applied into unstructured meshes.
