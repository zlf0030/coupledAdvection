\section{Introduction}
%The main method of interface capturing. VOF,front tracking, level set, coupled VOF and level set.
%1. The application of multi-phase flow in engineering and physics especially in nuclear safety.
%2. The importance of interface capture method in the simulation of multi-phase flow
%3. The variety of interface capture methods and the latest research on them.
% Interface capture and multiphase flow
Incompressible two phase flow with large density ratio at the free surface can be found in many natural phenomenon and industrial processes such as power plant, internal combustion, chemical reactor. Especially during nuclear reactor severe accidents, the molten corium from the fuel core region or the reactor pressure vessel (RPV) may fall into water pool. Then fuel coolant interaction (FCI) will happen and can lead to steam explosion which can cause severe damage to the containment of the nuclear reactor. Severity of steam explosion can be decided by the limited breakup of the molten pour stream flowing through the water prior to stream explosion\cite{Ginsberg1986Liquid}. In order to study the effect of melt physical properties on the jet breakup length and the characterization of droplet size distribution in the premixing region, it is crucial to track molten and water interface motion precisely.

%Main interface capture methods
Computing interface motion plays an important role in numerical multi-phase flows research. There are several methods developed for interface tracking or capturing in multi-phase Computational Fluid Dynamics (CFD) and different methods have their own characteristics. In general, the interface tracking method and interface capturing method are the two main classes of methods that are used to locate the interface. Both the classes are validated against a Taylor bubble benchmark problem by Marshall \textit{et al.}\cite{marschall2014validation}.

Interface tracking methods includes the arbitrary Lagrangian-Eulerian (ALE) method and the marker and cell (MAC) method. A typical ALE method is front tracking method \cite{Unverdi1992A,TRYGGVASON2001708}, which is based on an adaptive mesh that deforms with the interface that is advected in a Lagrangian fashion by the velocity interpolated from the CFD velocity field. MAC method use a series of points inside a fixed-grid region to represent the interface. The points are moved in Lagrangian fashion in the velocity field solved from Navier-Stokes equations. Interface tracking method can precisely describe the motion of interface but can not strictly conserve mass and the algorithm can be very sophisticated and time-consuming when the interface is deeply twisted. And this method is still under development and improvement. Mari{\'c} \textit{et al.} implemented a compromising hybrid Level Set/Front tracking Method for unstructured grids within OpenFOAM framework \citep{MARIC201520}.

The interface capturing methods as another approach to simulate free surface and interfacial flows mostly include the volume of fluid(VOF) method\cite{hirt1981volume,rudman1997volume} and the level set(LS) method\cite{Sussman1994A,chang1996level}, which both adopt a scalar function to capture the interface in fixed Eulerian grids. Currently the interfacial flow solvers take variants VOF methods as the interface advection step in most CFD codes, which include current versions of ANSYS Fluent$^{\textregistered}$, STAR-CCM++$^{\textregistered}$,OpenFOAM$^{\textregistered}$ and so on. The VOF method defines void fraction $\alpha$ as the fraction of volume occupied by the liquid in each cell. The void fraction $\alpha$ bounded between 0 and 1 at the cells fully occupied by one of phases, changes discontinuously across the interface. The most crucial step of the method is to solve the advection equation of $\alpha$. But due to the discontinuous nature of $\alpha$, large numerical diffusion in the convection scheme causes non-physical smearing of the interface. In order to describe the interface precisely, most VOF methods are separated into two categories, geometric VOF \cite{diot2016interface,hirt1981volume,Gueyffier1999Volume,NohAndWoodward1976,RIDER1998112,Youngs1982,roenby2016computational,puckett1997high} and algebraic VOF \cite{rudman1997volume,Muzaferijia1998,Ubbink1997,weller2008new,rusche2003computational,deshpande2012evaluating}. Geometric VOF methods define or reconstruct the interface location within a cell using the function value $\alpha$, while algebraic VOF methods use high solution schemes and compressive to reduce the numerical diffusion without defining exactly the location of interface.
%%\subsubsection{Geometrical VOF}

Generally, the interface described by the geometrical methods is more accurate and less smeared than the algebraic methods. Because in most geometrical methods, the interface segment is reconstructed within a cell volume by piecewise lines. During the past decades, scientists proposed variants geometrical methods, which include donor-acceptor\cite{hirt1981volume}, SLIC(simple line interface calculation)\cite{NohAndWoodward1976} and PLIC(piecewise linear interface calculation)\cite{Youngs1982}. The PLIC method is a second-order algorithm and the reconstructed interface is much shaper than other geometrical methods.The article\cite{rudman1997volume} summaries the three geometrical methods and compares them with FCT(flux-corrected transport)-VOF method proposed in the article. The conclusion is that Youngs' method\cite{Youngs1982} may be more accurate but more complicated to apply in three dimensions and unstructured meshes than FCT-VOF methods. To be applied for three dimensions, the piecewise-linear interface calculation method is proposed in \cite{Gueyffier1999Volume}. And the interface reconstruction method is realized in \cite{diot2016interface} for three dimensional arbitrary convex cells. One of the geometrical algorithms called isoAdvector is proposed in \cite{roenby2016computational} and implemented into OpenFOAM$^{\textregistered}$ as the incompressible two-phase flow solver, isoAdvector. This method uses the concept of isovalue contours to locate the interface in cells. For full details, the source code is provided with this website\cite{isoAdvector}.
%%\subsubsection{Algebraic VOF}

Among algebraic methods, using standard upwind scheme to transport the VOF field is the easiest one to apply. Most algebraic methods use the compressive algorithms to discretize the convective term in the VOF advection equation for preserving the interface sharpness. The interface typically spreads over a few cells and diffuses lowly but unboundedly under certain conditions\cite{Ubbink1997}. The interface smearing can be minimized and bounded by adding a compressive term or counter gradient term to the VOF advection equation\cite{weller2008new}. Some solvers like interFoam and multiphaseInterFoam in OpenFOAM$^{\textregistered}$ use the algebraic methods, also called Multidimensional Universal Limiter with Explicit Solution (MULES) algorithm, which is highly adaptive to three dimensions and unstructured grids. The performance of the two-phase flow solver - interFoam is evaluated in \cite{deshpande2012evaluating} and the basic principle of MULES algorithm can also be found.

%%\subsection{Level set methods}
%%Level set function.
 Level set method was first proposed by Osher and Sethian \cite{osher1988fronts}, and then it was coupled to the equations for two-phase incompressible flow by Sussman \textit{et al.}\cite{Sussman1994A}. LS method was proved to be a powerful algorithm to handle complex topological changes such as merging, twisting and pinching by following research \cite{Sussman1995A,chang1996level,sussman1998improved}. And exhaustive explanation and general applications of LS method can be found in Osher and Fedkiw's book\cite{osher2006level} and Sethian's book\cite{sethian1999level}. In stead of using bounded volume fraction, the idea of LS method is to represent the interface with the zero value iso-face of a level set function $\phi$. The value of $\phi$ is positive in one phase and negative in the other \cite{chopp1991computing}. One advantage of LS method is that the interface is advected implicitly by solving the advection equation of $\phi$, which solved algebraically with high order discretization schemes like WENO (weighted essentially non-oscillatory) scheme \cite{liu2017coupled,son2002coupled,martin2018implementation}.
%%Reinitialization function.
Due to the Lipschitz-continuous nature of the level set function, which is usually takes the form of the signed distance to the interface, the derivatives of $\phi$ are easy to calculate as well as the normal and curvature of the interface \cite{ge2018efficient}. The signed distance function prevents gradients of $\phi$ from being steep and flat as that can cause numerical instabilities and loss inaccuracy.  As $\phi$ is advected by solving the transport equation, the interface's shape is changed and the level set function loses the characteristic of the signed distance function that can be restored by reinitialization equation\cite{Sussman1994A,Johansson422383}. However, mass loss/gain always occurs along with the simulation of incompressible flows, because LS function can not provide volume information as VOF does. Furthermore, numerical errors arise from solving the LS advection equation and/or the reinitialization equation. Especially in areas of high curvature or other unsolved regions, the discretization of advection equation can cause inevitable and significant numerical dissipation\cite{losasso2006spatially}. Many attempts have been made to improve the mass conservation of the level set method and reduce numerical dissipation from advection equation and reinitialization process. This article \cite{luo2015mass} summarized several glories of improvement strategies.

A lot of studies were devoted to improve level set method. Some applied high solution schemes like fifth-order WENO scheme\cite{nourgaliev2005improving,salih2009some}, discontinuous Galerkin method \cite{reed1973triangular,rasetarinera2001efficient,remacle2007efficient} and semi-Lagrangian approach\cite{enright2005fast,strain1999semi,xiu2001semi}. Some articles studied velocity extensions method to maintain the signed distance function\cite{adalsteinsson1999fast,chopp2009another,ovsyannikov2012new,sabelnikov2014modified}. Moreover, using hyperbolic tangent function to supersede the signed distance function as the level set function was firstly proposed by Olsson \textit{et al.}\cite{OLSSON2005225,OLSSON2007785}. A series of work based on the method has been done\cite{chiodi2017reformulation,chiu2011conservative,sato2012conservative,sheu2009development,sheu2011development,NONOMURA201495,owkes2013discontinuous,xiao2005simple,walker2010conservative,desjardins2008accurate}. But small pieces of fluid are found breaking off from the interface and moving with erroneous velocity field\cite{luo2015mass}. Besides, Kohno \textit{et al.} \cite{KOHNO20131547} proposed a novel numerical method for solving the advection equation for a level set function by using hierarchical-gradient truncation and remapping (H-GTaR) of the original PDE. In addition, spatially adaptive level set methods were developed to improve the accuracy of the interface location. These methods include the adaptive level set approach\cite{sussman1999adaptive}, the octree based methods \cite{losasso2006spatially,fuster2009simulation}, the structured adaptive mesh refinement\cite{nourgaliev2005improving}, the refined level set grid (RLSG) method \cite{herrmann2008balanced}, the spectrally refined interface approach \cite{desjardins2009spectrally} and the adaptive level set method\cite{kim2011accurate}. These articles\cite{nguyen2001boundary,Gihun2005,GIBOU2007536,tanguy2005application} combined LS method with Ghost FLuid Method \cite{fedkiw1999non} and the boundary conditional capturing technology\cite{liu2000boundary}. On the other hand, Some studies are aimed for improving the accuracy and efficiency of the reinitialization process\cite{sussman1998improved,sussman1999efficient,RUSSO200051,HARTMANN20086821}. There are several ways to restore the signed distance function, including PDE (partial differential equation) method \cite{peng1999pde,chang1996level,MCCASLIN2014408} and FMM (fast marching method) \cite{adalsteinsson1995fast,sethian1996fast,sethian2001evolution,salac2011augmented,desjardins2009spectrally,desjardins2008accurate}. Besides, more sophisticated methods for reinitialization have been proposed, including geometric mass-preserving redistancing scheme \cite{ausas2011geometric} and the volume-reinitialization scheme\cite{salih2013mass}.Hysing and Turek discuss and compare these reinitialization methods in \cite{hysing2005eikonal}.

%(1) High solution schemes like fifth-order WENO scheme can improve the accuracy and reduce mass losses\cite{nourgaliev2005improving,salih2009some}. Discontinuous Galerkin method was created by Reed and Hill\cite{reed1973triangular}. These articles\cite{rasetarinera2001efficient,remacle2007efficient} applied this method in the discretization of the level set equation because of its high-order accurate compact scheme. In addition, the semi-Lagrangian approach has been applied to the discretization of the level set function by first-order schem\cite{enright2005fast,strain1999semi} and high-order scheme\cite{xiu2001semi}.

%(2) Velocity extensions are used to map the velocity of interface into the rest of computational domain\cite{adalsteinsson1999fast}, which is based on the fact that the gradient of the level set function grows rapidly in shearing velocity fields\cite{chopp2009another}. Extensive velocity method can maintain the signed distance function by introducing the source term into the advection equation. But first-order approximation still causes numerical artifacts which leads to higher-order schemes applied in the method\cite{ovsyannikov2012new,sabelnikov2014modified}.

%(3) Hyperbolic tangent function is used by THINC (tangent of hyperbola for interface capturing) scheme to supersede the signed distance function as the level set function, which is firstly proposed by Olsson \textit{et al.}\cite{OLSSON2005225,OLSSON2007785}.A lot of work based on the method has been done\cite{chiodi2017reformulation,sato2012conservative,sheu2009development,sheu2011development,NONOMURA201495,owkes2013discontinuous,xiao2005simple,walker2010conservative,desjardins2008accurate}. But small pieces of fluid are found breaking off from the interface and moving with erroneous velocity field\cite{luo2015mass}. Besides, Kohno \textit{et al.} \cite{KOHNO20131547} proposed a novel numerical method for solving the advection equation for a level set function by using hierarchical-gradient truncation and remapping (H-GTaR) of the original PDE.

  %*Chang \textit{et al.}\cite{chang1996level} proposed a re-initialization procedure involving a perturbed Hamilton-Jacobi equation to a steady state. *McCaslin and Desjardins \cite{MCCASLIN2014408} further improved the reinitialization equation to account for significant amount of spatial variability in level set transport. Another reinitialization method FMM proposed to reduce computational cost was extended to get higher-order accuracy\cite{desjardins2009spectrally,sethian1999level,desjardins2008accurate}. Other methods for reinitialization have been proposed, including geometric mass-preserving redistancing scheme \cite{ausas2011geometric} and the volume-reinitialization scheme\cite{salih2013mass}.

To take advantages of both LS and VOF methods, Sussman \textit{et al.}\cite{SUSSMAN2000301,SUSSMAN2003110} proposed a coupled level set and volume of fluid (CLSVOF) method. These studies applied CLSVOF method and its variants\cite{wang2012new,wang2010sharp,lv2010novel,menard2007coupling,yang2006adaptive,son2003efficient}.
Some improved CLSVOF methods, like VOSET method\cite{sun2010coupled},the conservation correction equation method\cite{kees2011conservative}, and the level set volume constraint method\cite{wang2012hybrid} were developed. Cao \textit{et al.}\cite{cao2018coupled} applied VOSET method based on multi-dimensional advection for unstructured triangular meshes. Pijl \textit{et al.} \cite{van2008computing,van2005mass} proposed a mass-conserving method, which does not need to construct the complicated interface. Besides, these articles\cite{le20133d,trontin2012subgrid,hieber2005lagrangian,enright2002hybrid} applied the hybrid Lagrangian-Eulerian method to rebuild the level set function.

%(6)Spatially adaptive level set methods were developed to improve the accuracy of the interface location. These methods include the adaptive level set approach\cite{sussman1999adaptive}, the octree based methods \cite{losasso2006spatially,fuster2009simulation}, the structured adaptive mesh refinement\cite{nourgaliev2005improving}, the refined level set grid (RLSG) method \cite{herrmann2008balanced}, the spectrally refined interface approach \cite{desjardins2009spectrally} and the adaptive level set method\cite{kim2011accurate}.

In this paper, we introduce the CLSAdvection algorithm which is a mass conservative level set method coupled with volume of fluid method. The purpose of developing this algorithm is to combine the continuous nature of level set function $\phi$ and the mass conservative property of void fraction $\alpha$. CLSAdvector algorithm significantly reduces the smearing of the interface and avoids the mass loss caused by numerical diffusion. This algorithm is developed into a multi-phase solver implemented in the OpenFOAM$^{\textregistered}$ CFD software.

In the reminder of this section, we review the interface problem and all kinds of previous studies on the interface tracking/captring methods. In section 2, we introduce the concept of CLSAdvector algorithm and give an overview of the steps involved in the numerical procedure. In section 3, the corresponding numerical implementations are presented in detail. This is followed by section 4, in which four benchmark case are tested for validation. Finally, some conclusions are drawn in section 5.
